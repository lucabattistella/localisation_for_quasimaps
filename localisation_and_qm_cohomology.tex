\documentclass[11pt]{amsart}
%\usepackage[english]{babel}
\usepackage{appendix}
\usepackage{amsmath}
\usepackage{amsfonts}
\usepackage{amssymb}
%\usepackage{showlabels}
\usepackage{hyperref}
\usepackage{amsthm}
\usepackage{marginnote}
\usepackage{stmaryrd}
\usepackage{enumitem}
\usepackage[english]{babel}
\usepackage{yfonts}
\usepackage[T1]{fontenc}
\usepackage[utf8x]{inputenc}
%\usepackage{enumerate}
\usepackage{verbatim}
\usepackage{graphicx}
\usepackage{verbatim}
\usepackage{faktor}
\usepackage{xcolor}
\usepackage{xfrac}
\usepackage{tikz,tikz-cd}
\usetikzlibrary{decorations.pathmorphing,decorations.pathreplacing,patterns}
\usepackage[all]{xy}
\usepackage{bbm}
\usepackage{tabularx}
\usepackage{longtable}
\usepackage{tabu}
\usepackage{booktabs}
\usepackage{mathtools}

\newcommand{\TT}{\operatorname{T}}
\newcommand{\M}[4]{\overline{\mathcal{M}}_{#1,#2}(#3,#4)}
\newcommand{\Q}[4]{\mathcal{Q}_{#1,#2}(#3,#4)}
\newcommand{\Qe}[4]{\mathcal{Q}^{\epsilon}_{#1,#2}(#3,#4)}
\newcommand{\Qt}[4]{\widetilde{\mathcal Q}_{#1,#2}(#3,#4)}
\newcommand{\QG}[4]{\mathcal{Q}G_{#1,#2}(#3,#4)}
\newcommand{\QGe}[4]{\mathcal{Q}G^{\epsilon}_{#1,#2}(#3,#4)}
\newcommand{\D}[3]{\mathcal{D^Q}(#1,#2,#3)}
\newcommand{\E}[3]{\mathcal{E^Q}(#1,#2,#3)}
\newcommand{\PP}{\mathbb P}
\newcommand{\Z}{\mathbb{Z}}
\newcommand{\N}{\mathbb{N}}
\newcommand{\OO}{\mathcal{O}}
\renewcommand{\to}{\rightarrow}
\newcommand{\A}{\mathcal A}
\newcommand{\B}{\mathcal B}
\newcommand{\C}{\mathfrak C}
\newcommand{\EE}{\mathbf{E}}
\renewcommand{\L}{\mathcal L}
\newcommand{\LL}{\mathbf{L}}
\newcommand{\MM}{\mathfrak M}
\newcommand{\Aaff}{\mathbb{A}}
\newcommand{\kfield}{\Bbbk}
\newcommand{\comp}{\chi}
\newcommand{\sst}{\sigma^{\operatorname{ss}}}
\newcommand{\Pic}{\operatorname{Pic}}
\newcommand{\Def}{\operatorname{Def}}
\newcommand{\Spec}{\operatorname{Spec}}
\newcommand{\Proj}{\operatorname{Proj}}
\newcommand{\Hom}{\operatorname{Hom}}
\newcommand{\Ext}{\operatorname{Ext}}
\newcommand{\val}{\operatorname{val}}
\newcommand{\Gm}{\mathbb{G}_{\text{m}}}
\newcommand{\virt}[1]{[#1]^{\operatorname{virt}}}
\newcommand{\vip}[1]{[#1]^{\operatorname{prod}}}
\newcommand{\Id}{\operatorname{Id}}
\newcommand{\CC}{\mathbb{C}}
\newcommand{\QQ}{\mathbb{Q}}
\newcommand{\ZZ}{\mathbb{Z}}
\newcommand{\HH}{\operatorname{H}}
\newcommand{\Achow}{\operatorname{A}}
\newcommand{\pt}{\operatorname{pt}}
\newcommand{\bq}{\begin{equation}}
\newcommand{\eq}{\end{equation}}
\newcommand{\ba}{\begin{aligned}}
\newcommand{\ea}{\end{aligned}}
\newcommand{\be}{\begin{enumerate}}
\newcommand{\ee}{\end{enumerate}}
\newcommand{\bsm}{\left(\begin{smallmatrix}}
\newcommand{\esm}{\end{smallmatrix}\right)}                   
\newcommand{\bpm}{\begin{pmatrix}}
\newcommand{\epm}{\end{pmatrix}}
\newcommand{\barr}{\begin{displaymath}\begin{array}{cccc}}
\newcommand{\earr}{\end{array}\end{displaymath}}
\newcommand{\barrl}{\begin{displaymath}\begin{array}{lcl}}
\newcommand{\earrl}{\end{array}\end{displaymath}}
\newcommand{\barl}{\begin{displaymath}\begin{array}{l}}
\newcommand{\earl}{\end{array}\end{displaymath}}
\newcommand{\bxym}{ \begin{displaymath}\xymatrix }
\newcommand{\exym}{\end{displaymath}}
\newcommand{\bcd}{\begin{center}\begin{tikzcd}}
\newcommand{\ecd}{\end{tikzcd}\end{center}}
\newcommand{\R}{\operatorname{R}^{\bullet}}
%\newcommand{\sslash}{\mathbin{/\mkern-6mu/}}
\newcommand{\tr}{{\rm tr}}
\newcommand{\Isom}{\text{Isom}}
\newcommand{\pr}{\operatorname{pr}}
\newcommand{\ev}{\operatorname{ev}}
\newcommand{\codim}{\operatorname{codim}}
\newcommand{\vdim}{\operatorname{vdim}}
\newcommand{\ildef}[1]{\emph{#1}}
\newcommand{\om}[1]{\mathcal{#1}}
\newcommand{\h}{\operatorname{h}}
\newcommand{\Aut}{\operatorname{Aut}}
\newcommand{\RR}{\textbf{R}}
\newcommand{\NN}{\operatorname{N}}

\theoremstyle{definition}
\newtheorem{thm}{Theorem}[section]
\newtheorem{lem}[thm]{Lemma}
\newtheorem{lemma}[thm]{Lemma}
\newtheorem{prop}[thm]{Proposition}
\newtheorem{cor}[thm]{Corollary}
\newtheorem*{teo*}{Theorem}
\newtheorem{ipotesi}{ipotesi}
\newtheorem*{nota}{Nota}
\newtheorem{claim}{Claim}
\newtheorem{question}[thm]{Question}
\newtheorem{conj}[thm]{Conjecture}

\newtheorem{innercustomthm}{Theorem}
\newenvironment{customthm}[1]
  {\renewcommand\theinnercustomthm{#1}\innercustomthm}
  {\endinnercustomthm}

\theoremstyle{definition}
\newtheorem{example}[thm]{Example}
\newtheorem{ex}[thm]{Example}
\newtheorem{dfn}[thm]{Definition}
\newtheorem{definition}[thm]{Definition}
\newtheorem{aside}[thm]{Aside}
\newtheorem{remark}[thm]{Remark}
\newtheorem{com}[thm]{Comment}
\newtheorem{num}{Number}
\newtheorem*{sketch}{Sketch}
\newtheorem*{rem}{Remark}
\newtheorem*{aside*}{Aside}
\newtheorem*{acknowledgements}{Acknowledgements}

\newcommand{\ilemph}[1]{\emph{#1}}

\setcounter{tocdepth}{1}

\newcommand{\todo}[1]{\vspace{5mm}\par \noindent
\framebox{\begin{minipage}[c]{0.95 \textwidth} \tt #1\end{minipage}} \vspace{5mm} \par}

\def\ti{-\allowhyphens}
\newcommand{\thismonth}{\ifcase\month % case 0 --- impossible!
  \or January\or February\or March\or April\or May\or June%
  \or July\or August\or September\or October\or November%
  \or December\fi}
\newcommand{\thismonthyear}{{\thismonth} {\number\year}}
\newcommand{\thisdaymonthyear}{{\number\day} {\thismonth} {\number\year}}

\usepackage[T1]{fontenc}
\usepackage{newpxtext,newpxmath}

\title[]{Localisation and quasimap cohomology}
\author{}
\begin{document}

\maketitle
\appendixtitletocoff
\tableofcontents

\section{Localisation formula}
\subsection{Notation from toric geometry}
Let $X_\Sigma$ be a smooth complete toric variety, for $\Sigma\subseteq N$ a rational polyhedral fan and $M=\Hom_{\ZZ}(N,\ZZ)$. Let us denote by $r$ the Picard rank of $X$, by $n$ its dimension, and by $N=n+r$ the number of rays in $\Sigma$. Let $v_\rho$ denote the primitive generator of the ray $\rho$, and assume that $\Sigma(1)$ is an ordered set.

The (non-effective) action of the big torus $T=\Gm^N$ on $X$ induces an action on $\Q{g}{n}{X}{\beta}$, by scaling the sections. Let us denote by $\lambda_1,\ldots,\lambda_N$ the corresponding weights.

Let us denote by $\{\sigma_i\}_{i\in\Sigma^\text{max}}$ the $T$-fixed points on $X$ (corresponding to maximal cones of $\Sigma$) and by $\{\tau_{i,j}\}_{i,j\in\Sigma^\text{max}}$ the $1$-dimensional orbits (corresponding to facets of the maximal cones; $\tau_{i,j}$ -if it exists- connects $\sigma_i$ and $\sigma_j$).

Let $\sigma_i$ and $\sigma_j$ be two adjacent maximal cones; since $X$ is smooth, $\{v_\rho\}_{\rho\prec\tau_{i,j}}\cup\{v_n\}$ is a $\ZZ$-basis of $N$ (where $v_n$ is the only ray in $\sigma_i$, but not in $\tau_{i,j}$), so we can find the dual basis $\{m_1,\ldots,m_n\}$ of $M$. Define $\lambda^{\sigma_i}_{\sigma_j}=\sum_{\rho\in\Sigma(1)}\langle m_n,v_\rho\rangle\lambda_\rho$. Compare with \cite[\S\S 6.4 and 7.3]{HolgerSpielberg}.

\begin{lem}
 Let $\sigma_i$ be a $T$-fixed point on $X$ and $\tau_{i,j}$ be a $1$-dimensional orbit through it, furthermore let $D_\rho$ be a toric divisor. Then the weight of the $T$-action on $\mathcal O(D_\rho)_{\sigma_i}$ is
 \[
  \begin{cases}
      \lambda^{\sigma_i}_{\sigma_j}, & \text{if}\ \rho\prec \sigma_i\  \text{and}\  \tau_{i,j}\cup\{v_\rho\}=\sigma_i \\
      0, & \text{otherwise.}
    \end{cases}
 \]
The weight of the $T$-action on $T(\tau_{i,j})_{\sigma_i}$ is $\lambda^{\sigma_i}_{\sigma_j}$.
\end{lem}
\begin{proof}
 Let $\sigma_i$ be spanned by $\{v_{i_1},\ldots,v_{i_n}\}$. If $[z_1:\ldots:z_N]$ are homogeneous coordinates on $X$, then local coordinates around $\sigma_i$ are given by \[\left(x_{i_1}=z_{i_1}\prod_{j\neq i_1}z_j^{\langle m_{i_1},v_j\rangle},\ldots,x_{i_n}=z_{i_n}\prod_{j\neq i_n}z_j^{\langle m_{i_n},v_j\rangle}\right),\] where $\{m_{i_1},\ldots,m_{i_n}\}$ is the dual basis of $\{v_{i_1},\ldots,v_{i_n}\}$.
 
 If $\rho\nprec \sigma_i$ then the weight is $0$ because we can find a divisor representing $\mathcal O(D_\rho)$ that does not pass through $\sigma_i$. Otherwise $\rho=i_j$ for some $j\in\{1,\ldots,n\}$, so $D_{\rho}$ has local equation $x_{i_j}=0$ near $\sigma_i$, which makes the first statement clear.
 
 The second part follows from the exact sequence
 \[
  0\to T\tau_{i,j}\to TX_{|\tau_{i,j}}\to \bigoplus_{\rho\prec\tau_{i,j}}\mathcal O_{\tau_{i,j}}(D_{\rho})\to 0
 \]
together with the Euler exact sequence for $TX$ and the first part.
\end{proof}

\section{$T$-fixed loci}
The following discussion is inspired by \cite[\S 7.3]{MOP}. $T$-fixed loci for $\Q{g}{n}{X}{\beta}$ are indexed by decorated graphs
\[ \left(\Gamma, v, \gamma, b,\varepsilon,\delta,\mu\right) \]
where:
\begin{enumerate}
 \item $\Gamma=(V,E)$ is a graph with vertex set $V$ and edge set $E$ (no self-edges allowed);
 \item $v\colon V\to \{\sigma_i\}_{i\in\Sigma^\text{max}}$ assigns a fix point to each vertex;
 \item $\gamma\colon V\to \ZZ_{\geq 0}$ is a genus assignment;
 \item $b\colon V\to H^+_2(X,\ZZ)$ assigns an effective curve class to each vertex;
 \item $\varepsilon\colon E\to \{\tau_{i,j}\}_{i,j\in\Sigma^\text{max}}$ assigns a one-dimensional orbit to each edge;
 \item $\delta\colon E\to \ZZ_{\geq1}$ specifies the degree of the covering map;
 \item $\mu\colon V\to 2^{\{1,\ldots,n\}}$ is a distribution of the markings to the vertices $V$.
\end{enumerate}
These data are required to satisfy a number of compatibility conditions:
\begin{itemize}
 \item $\Gamma$ must be connected;
 \item if $e\colon v_1\to v_2$ then $\varepsilon(e)=\tau_{i,j}$ with $v(v_1)=\sigma_i$ and $v(v_2)=\sigma_j$ (or viceversa);
 \item $h^1(\Gamma)+\sum_{v\in V} \gamma(v)=g$;
 \item $b$ is compatible with $v$; \marginpar{CHECK ME AND SPELL ME OUT}
 \item $\sum_{v\in V}b(v)+\sum_{e\in E}\delta(e)[\varepsilon(e)]=\beta$.
 \end{itemize}
We are going to denote by $\val\colon V\to\ZZ_{\geq1}$ the number of edges adjacent to a vertex, and by $\deg\colon V\to\ZZ_{\geq2}$ the sum of $\val$ with the number of marked points associated to each vertex.

The corresponding $T$-fixed locus is isomorphic, up to a finite map, to:
\[
 \prod_{v\in V} \overline{\mathcal M}_{\gamma(v),\deg(v)|\sum_{\rho\nprec v(v)}b(v)\cdot D_\rho}
\]
The moduli spaces corresponding to degenerate vertices (where $\deg(v)=2$, $\gamma(v)=0$, and $b(v)=0$) are treated as points in this product. Notice that $\overline{\mathcal M}_{g,n|d}/S_d\simeq\Q{g}{n}{\Aaff^1\sslash\Gm}{d}$. Hence the finite map has degree \[|\mathbf A|\cdot\prod_{v\in V}(b(v)\cdot D_\rho)!\] where $|\mathbf A|$ can be extrapolated from
\[
 0\to\prod_{e\in E}\ZZ/\delta(e)\ZZ\to \mathbf A\to \Aut(\Gamma)\to 0.
\]
Notice here that, for every maximal cone $\sigma_i$, the collection $\{D_\rho\}_{\rho\nprec\sigma_i}$ constitutes a basis of $\Pic(X)$ (since every support function can be made into vanishing on every $\rho\nprec\sigma_i$ by subtracting an appropriate $m\in M$).

The corresponding quasimap can be described as follows: edges correspond to maps (without basepoints) from $\PP^1$ to the corresponding $1$-dimensional $T$-orbit $\varepsilon(e)$, of degree $\delta(e)$ and totally ramified at the two $T$-fixed points. Pick instead a vertex $v\in V$: according to $v(v)=\sigma_i$, we may write $\OO(D_{i_j})=\bigotimes_{\rho\nprec\sigma_i} \OO(D_\rho)^{\otimes a_{i_j,\rho}}$ for each $i_j,\ j=1,\ldots,n$ such that the corresponding ray belongs to $\sigma_i$. For a marked curve $C_v$ in the mixed moduli space $\overline{\mathcal M}_{\gamma(v),\deg(v)|\sum_{\rho\nprec v(v)}b(v)\cdot D_\rho}$ with markings
\[
 \{p_1,\ldots,p_{\deg(v)}\}\cup\bigcup_{\rho\nprec\sigma_i}\{q_{\rho,1},\ldots,q_{\rho,b(v)\cdot D_{\rho}}\}
\]
the corresponding quasimap is given by:
\[
 \left((C_v,\{p_1,\ldots,p_{\deg(v)}\}),(\mathcal O_{C_v}\hookrightarrow\OO_{C_v}(\sum_{j=1}^{b(v)\cdot D_{\rho} }q_{\rho,j})=:L_\rho)_{\rho\nprec\sigma_i},(\mathcal O_{C_v}\xrightarrow{0}\bigotimes_{\rho\nprec\sigma_i} L_\rho^{\otimes a_{i_j,\rho}}=:L_{i_j})_{j=1,\ldots,n} \right)
\]

Gluing along flags $F=(e,v)$ is made possible by the required compatibilities.

\appendix

\bibliographystyle{alpha}
\bibliography{relqm}

\bigskip\bigskip

%\noindent Luca Battistella\\
%Department of Mathematics, Imperial College London \\
%\texttt{l.battistella14@imperial.ac.uk}\\

%\noindent Navid Nabijou \\
%Department of Mathematics, Imperial College London \\
%\texttt{navid.nabijou09@imperial.ac.uk}



\end{document}